\documentclass{article}
\usepackage[utf8]{inputenc}

\title{Ohrtemperatur (Arbeitstitel)}
\author{jolly}
\date{\today}


\begin{document}

\maketitle
\newpage
\tableofcontents

\newpage
\section{Introduktion und Abstrakt}
Ziel dieser Arbeit ist es, mittels kontinulierlichen Messen den Unterschied der Temperaturen meiner Ohren zu  ermitteln und langfristig eine Statisitk zu erstellen, mit der trotz Messfehler eine durchschnittlich akurate Körpertempratur ermittelt werden kann. \\ Gemessen wird dabei mit einem Fieberthermometer der Marke Beurer, dem Beurer FT 58 Ohrthermometer. \footnote{https://www.beurer.com/web/de/produkte/medical/fieberthermometer/ft-58.php}

\newpage
\section{Messwerte}
Gemessen wurde über einen Zeitraum von n Tagen.
In der unten daregstellten Tabelle findet man das Datum, die Uhrzeit und den jeweils gemessenen Wert für das linke und das rechte Ohr.

\begin{table}[h!]
    \centering
    \begin{tabular}{c|c|c|c|c}
   $Datum $& $Uhrzeit $& $linkes Ohr $& $rechtes Ohr $ \\ \hline
    $17.01.2022$   & $00:24$ & $36,7$ &  $36,1$\\ \hline
    $17.01.2022$   & $08:06$ & $35,9$ &  $36,7$\\ \hline
    $17.01.2022$   & $10:34$ & $36,7$ &  $37,2$\\ \hline
    \end{tabular}
    \caption{Messwerte des Thermometers}
\end{table}

\newpage
\section{Auswertung}

\newpage
\section{Conclusio}

\end{document}
